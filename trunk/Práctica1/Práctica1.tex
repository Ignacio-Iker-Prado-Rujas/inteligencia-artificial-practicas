

\documentclass[spanish, a4paper, 12pt]{article} 	%idioma, tamaño, tamaño letra, tipo de documento (articulo, libro, report)
\usepackage[english, activeacute]{babel}		%babel para tildes: á = \´{a}	
\usepackage[utf8]{inputenc}
\usepackage{geometry}
\usepackage{multicol}
\usepackage{amsmath}
\usepackage{amssymb}
%\usepackage{amsttthm}
\usepackage{graphics}
\usepackage{graphicx}

\usepackage{fancyhdr}
\geometry{a4paper, textwidth=16cm, textheight=20cm}
\pagestyle{fancy}
%\usepackage{nicefrac}
%\setlenght{\parskip}{1}
\lhead{Métodos algorítmicos}
\cfoot{\thepage}
\renewcommand{\headrulewidth}{0.4pt}
\renewcommand{\footrulewidth}{0.4pt}

\begin{document}
\title{\textbf{Inteligencia Artificial}}
\maketitle


\center{\includegraphics[width=\textwidth]{robotHonda.png}}

\newpage
\textbf{{¿Qué es la inteligencia artificial?}}

\begin{section}{Deep Blue y Walter}
	Deep blue es una supercomputadora fabricada en los años 60 por IBM desarrollada para jugar al ajedrez, que en 1996 se enfrentó al campeón del mundo Gary Kaspárov.
\end{section}

\end{document}

