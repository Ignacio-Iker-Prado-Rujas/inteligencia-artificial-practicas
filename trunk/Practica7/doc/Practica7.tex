F\documentclass[11pt, a4paper, spanish, openright, twoside]{book}
\usepackage[spanish, activeacute]{babel}
\usepackage[utf8]{inputenc}
%\usepackage[top=2.5cm, bottom=2.5cm, outer=1.75cm, inner=1.75cm, heightrounded, marginparwidth=2.5cm, marginparsep=0.3cm]{geometry}	%márgenes empequeñecidos
\usepackage[top=2.95cm, bottom=2.25cm, outer=2.75cm, inner=2.75cm, heightrounded, marginparwidth=2.5cm, marginparsep=0.3cm]{geometry}	%márgenes originalmente
\usepackage{dpg}
\usepackage{fli}

\usepackage{pgf}
\usepackage{tikz}

\usepgflibrary{shapes.geometric} % LATEX and plain TEX and pure pgf
\usetikzlibrary{arrows,automata,positioning}
\tikzstyle{accepting by double}= [double distance=1.6pt,double,outer sep=.5\pgflinewidth+.8pt] % esto es algo estético.
\renewcommand\shorthandsspanish{}  % para compatibilizar spanish con tikz

%%%%%%		Figuras		%%%%%%%%%%%%%%%%%%%
\usepackage[vflt]{floatflt}		%Entorno float-figure

%%%%%%		Page style		%%%%%%%%%%%%%%%%%%%
\renewcommand{\thepage}{\arabic{page}}% Arabic page numbers\fancyhead{}
\pagestyle{fancy}
\fancyfoot{}
\fancyhead[LO,RE]{Práctica 6}	%encabezado de pares: nombre de la sección
\fancyhead[RO,LE]{Diseño de un agente de viajes}
\fancyfoot[LE,RO]{\thepage}	%abajo a izqda en pares, derecha en impares: numero de pagina
%\fancyhead[LE]{\nouppercase{\leftmark}} %cuadro izquierdo de pagina par: parte y contador
\fancyfoot[CE]{Inteligencia Artificial} 
\fancyfoot[CO]{Doble Grado Informática-Matemáticas - Universidad Complutense}
\renewcommand{\footrulewidth}{0.4pt}
\renewcommand{\headrulewidth}{0.4pt}		% linea por debajo del encabezado
\renewcommand{\sectionmark}[1]{\markright{\textbf{\thesection. #1}}}	%negrita
\renewcommand{\labelitemi}{$\circ$} %Primer itemize con circunferencia vacia
\renewcommand{\labelitemii}{$\cdot$} %Segundo itemize con punto pequeño \cdot
\renewcommand*{\thesection}{\arabic{section}}	% Hace que no apareca el indice de capitulos y que comience en section

%%%%%%		Others		%%%%%%%%%%%%%%%%%%%
\setlength{\leftmarginii}{0em} %Segundo itemize sin sangria
\setlength{\leftmarginiii}{1em} %Tercer itemize casi sin sangria
\renewcommand{\labelitemiii}{ }
\pagenumbering{roman}
\addto{\captionsspanish}{\renewcommand*{\contentsname}{Índice}} %Cambia "Indice general" por "Indice"



\begin{document} 
\title{\Huge{\textsc{Inteligencia Artificial}} \\
	\vspace{0.7cm}
	 \textsc{\Large{Práctica 7}} \\
	\vspace{1.5cm}
	\includegraphics[scale=0.45]{jarras}}
\author{\textsc{Grupo 3:}\\
	Enrique Ballesteros Horcajo\\
	Ignacio Iker Prado Rujas}
\date{\Today}
\maketitle

\newpage
\mbox{}
\thispagestyle{empty}						% Hoja en blanco, sin numeros ni nada
\newpage


\tableofcontents 							%INDICE hipervinculado

\newpage
\mbox{}
\thispagestyle{empty}						% Hoja en blanco, sin numeros ni nada
\newpage

\pagenumbering{arabic}						% Pone el contador de paginas a 1 y ahora en numeros normales

\vspace{3cm}


\newpage



\begin{section}{Diferencias entre Prolog y Jess}
		En Jess se define un espacio concreto para los hecho iniciales (deffacts ini), en cambio en Prolog no es necesario la parte procedimental. Eso nos permite que 
		al declarar los hechos iniciales al principio sean los que primero se añadan.
		Por tanto en Jess distinguimos explícitamente las reglas de los hechos, utilizando el deffrule y dándole un nombre concreto a la regla que puede ser distinto del assert que se cree. En prolog las reglas 
		y los hecho no se diferencian explícitamente.
		Por supuesto la sintaxis concreta es diferente: Uso de "\=" para el distinto, de "_",  en Prolog en vez de sus correspondientes en Jess.
		
		Una de las diferencias importantes está en la recursión, que en Prolog requiere colocar la recursión en la parte derecha de la sentencia y en Jess no es estrictamente necesario.
		

	
\end{section}
	\newpage
	\begin{section}{Algoritmo usado por Prolog}

		Sabemos que Prolog ejecuta de manera procedimental, por lo que ejecuta las operaciones en el orden que están escritas. Cambiando de orden estas operaciones obtenemos primero las siguientes soluciones:

		\begin{subsection}{Ejecuciones con órdenes distintos}
			Para el orden:  llenar3, llenar4, verter3, verter4, vaciar3, vaciar4.

			[llenar3,llenar4,vaciar3,verter4,vaciar3,verter4,llenar4,verter4] 8
			[llenar3,llenar4,vaciar3,verter4,vaciar3,verter4,llenar4,verter4,vaciar3] 9
			[llenar3,llenar4,vaciar3,verter4,vaciar3,verter4,verter3,verter4,llenar4,verter4] 10
			
			Encuentra la solución óptima a la decimo séptima.

			Para el orden: llenar3, verter3, vaciar4, llenar4, verter4 vaciar3.(Intentamos obtener cuanto antes una solución óptima)

			Encuentra una solución óptima a la quinta:
			[llenar3,verter3,llenar3,verter3,vaciar4,verter3]  6
			
		\end{subsection}

		Así, observamos que explora en profundidad, pues las soluciones que va encontrando de manera consecutiva tienen las primeras operaciones idénticas. Además, 
		sigue el orden de exploración en el que colocamos la deficición de las operaciones. Es un método no informado, pues no utilizamos ninguna heurística. Además, la profundidad no está acotada 
		pues explora hasta haber visitado todos los estados posibles incluso a partir de soluciones encontradas. Lo único que controlamos es que no se repitan estados, pues tendríamos una secuencia infinita 
		en muchos casos (por ejemplo si las primeras dos operaciones fueses llenar3 y vaciar3, sin control de repeticiones se quedaría alternando entre ambas hasta infinito. Es decir, que sin este control sería una búsqueda 
		primero en profundidad, pues no encuentra la solución óptima y puede entrar en caminos infinitos. Luego como controlamos las repeticiones es "primero en profundidad + control de repeticiones".

	
	\end{section}

	
\begin{thebibliography}{9}

\bibitem{aima}
	Russell, S.; Norvig, P, \\
	\emph{Artificial Intelligence, a modern aproach}.\\
	New Jersey: Pearson, 2010.
	
\bibitem{clase}
	Apuntes y transparencias de Inteligencia Artificial, \\
	Doble Grado Matemáticas - Ing. Informática, U.C.M., 2014-2015.

\end{thebibliography}


\end{document}