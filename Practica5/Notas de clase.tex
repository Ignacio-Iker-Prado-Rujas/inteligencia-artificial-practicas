\documentclass[11pt, a4paper, spanish, openright, twoside]{book}
\usepackage[spanish, activeacute]{babel}
\usepackage[utf8]{inputenc}
%\usepackage[top=2.5cm, bottom=2.5cm, outer=1.75cm, inner=1.75cm, heightrounded, marginparwidth=2.5cm, marginparsep=0.3cm]{geometry}	%márgenes empequeñecidos
\usepackage[top=2.95cm, bottom=2.25cm, outer=2.75cm, inner=2.75cm, heightrounded, marginparwidth=2.5cm, marginparsep=0.3cm]{geometry}	%márgenes originalmente


\begin{document}


	\begin{section}{Introducción}
		Jess:


			Vamos a separar el motor de inferencia ( que nos lo dan hecho) de las reglas(conocimiento) y vamos a construir 					nuestra base de reglas usando la sintaxis de Jess.

			deffacts: añadir los hechos iniciales. Se cargan cuando se inicializa el sistema.
			asserst: añadir hechos durante la ejecución.
			
			
				
					Cliente
				
			Nombre
			edad
			
			Museos
			Playa
			Viaje:
			presupuesto
			nº personas . amigos, familia niños
			descanso
			dias
			
			
				Destinos: Deben poder obtenerse a partir de bases de datos
				
			Ibiza:
				500 hoteles
				1000 apartaoteles
				50 discotecas
				1 teatro
				10 centros comerciales
				ESPAÑA
				compras
				ocio nocturno
				
				
				Clasificacion de Destinos:
					compras-> nº centros comerciales >5
					ocio nocturno-> nº discotecas > 10
					gastronomia -> nº restaurante > 50
					nacional -> pais = ESPAÑA
				
				Clasificacion de clientes:
					jovenes-> edad < 30
					medio-> edad > 30 y edad < 65
					mayores-> edad > 65
					aventura->no hay por qué preguntarle al usuario
					fiesta-> joven y aventura -> fiesta (por ejeplo)
					
				Alojamiento.>
					camping -> aventura, joven y barato
					hotel
					
				
					
		Debo asociar a un cliente con un destino, un alojamiento y un transporte.
		
			
				Prototipo 0:
					Campo: sí o no (podemos tener el campo vacío)
					Ópera: sí o no.
					Naturaleza sí.
					Primero usar ENUM y luego pasar a MULTISLOT.
		
		puedo asertar:(tipo pepe aventura) 
		rellenar un campo de la plantilla cuando lo conozca
				
	\end{section}

\end{document}